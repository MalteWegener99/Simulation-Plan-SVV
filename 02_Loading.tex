\section{Description of loading cases}
\label{sec:loading}
%We need a little more explanations here, I think. Let's try to expand it a bit
\subsection{Loading scenario}
The concerned loading case is caused by the critical loading scenario.
This loading scenario is described as a maximum upward deflection of the aileron. Furthermore the aerodynamic loading on the wing is at limit load and one of the two actuators is jammed.


\subsection{Geometry and Free body diagrams}
The geometry of the aileron, along with the dimensions, is visualized in \ref{fig:Dimensions_top_view}, \ref{fig:Dimensions_side_view} and \ref{fig:Dimensions_back_view}.

The loads acting on the aileron in the relevant loading case are visualized in the free body diagrams for the top view, side view and back view. These can be seen in \ref{fig:FBD_top_view}, \ref{fig:FBD_side_view} and \ref{fig:FBD_back_view}
The 
\par
\begin{figure}[H]
\begin{minipage}[b]{0.45\linewidth}
\centering
\includegraphics[width=8cm]{Images/FBD_top_view.JPG}
\caption{Free body diagram: Top view}
\label{fig:FBD_top_view}
\end{minipage}
\hspace{0.5cm}
\begin{minipage}[b]{0.45\linewidth}
\centering
\includegraphics[width=8cm]{Images/FBD_side_view.JPG}
\caption{Free body diagram: Side view}
\label{fig:FBD_side_view}
\end{minipage}
\begin{minipage}[b]{\linewidth}
\centering
\includegraphics[width=8cm]{Images/FBD_back_view.JPG}
\caption{Free body diagram: Back view}
\label{fig:FBD_back_view}
\end{minipage}
\end{figure}

\begin{figure}[H]
\begin{minipage}[b]{0.45\linewidth}
\centering
\includegraphics[width=8cm]{Images/Dimension_top_view.jpg}
\caption{Geometry and dimensions: Top view \cite{Assignment_description}.}
\label{fig:Dimensions_top_view}
\end{minipage}
\hspace{0.5cm}
\begin{minipage}[b]{0.45\linewidth}
\centering
\includegraphics[width=8cm]{Images/Dimension_side_view.jpg}
\caption{Geometry and dimensions: Side view \cite{Assignment_description}. }
\label{fig:Dimensions_side_view}
\end{minipage}
\begin{minipage}[b]{\linewidth}
\centering
\includegraphics[width=8cm]{Images/Dimension_back_view.jpg}
\caption{Geometry of the deformed and undeformed aileron : Back view \cite{Assignment_description}.}
\label{fig:Dimensions_back_view}
\end{minipage}
\end{figure}

\begin{figure}[H]
    \centering
    \includegraphics[width=15cm]{Dimensions_table.jpg}
    \caption{Parameters of aileron of the a Fokker 100}
    \label{fig:Parameters_aileron_F100}
\end{figure}

\subsection{Reference frame}
The reference frame used for this model has its x-axis in span wise direction, pointing towards the tip of the undeformed main wing. The z-axis is in chord wise direction, pointing towards the leading edge. Finally, the y-axis is directed upwards, orthogonal to the aileron surface. This reference frame is visualised in the free body diagrams in Figures \ref{fig:FBD_top_view}, \ref{fig:FBD_side_view} and \ref{fig:FBD_side_view}.

\subsection{Loading case}
The locations of the three hinges and the two actuators can be seen in Fig. \ref{fig:Dimensions_top_view}
The aileron is attached to the wing via three hinges, numerically ordered from right to left in Fig. \ref{fig:Dimensions_top_view}. The second (middle) hinge is fixed in x-,y- and z-direction.
The first (right) and second (left) hinge are fixed only in y-direction.
The actuator that is numbered I in Fig. \ref{fig:Dimensions_top_view} is fixed in z-direction.
At each hinge there are three reaction forces in x-,y- and z-direction.
\par
A point load P acts at actuator II in negative y- and z-direction. Its direction is equal to a negative rotation around the x-axis that is equal to the maximum upward deflection angle. 
A distributed load q is present due to aerodynamic forces. Its direction is in negative y-direction, perpendicular to the symmetry plane of the aileron (x-,z-plane). The magnitude of the distributed load varies over the aileron surface in both x- and z-direction. 